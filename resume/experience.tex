%-------------------------------------------------------------------------------
%	SECTION TITLE
%-------------------------------------------------------------------------------
\cvsection{项目经历}


%-------------------------------------------------------------------------------
%	CONTENT
%-------------------------------------------------------------------------------
\begin{cventries}

%---------------------------------------------------------
  \cventryjob
    {} % Job title
    {大杂烩项目} % Organization
    {1970.04-1970.04} % Location
    {} % Date(s)
    {
      \begin{cvitems} % Description(s) of tasks/responsibilities
        \item {\textbf{项目简介}:
        Rust是一种多范式、通用的编程语言。Rust 强调性能、类型安全和并发性,大大增加了项目的 Bigger 程度。}
        \item {\textbf{技术选型}:Rust、\textbf{React}、MyBatis-Plus、\textbf{Ruby}、Oracle、Redis、\textbf{Docker}、Maven、Jenkins、CheckStyle}
        \item {负责xxx功能模块的开发和实现。}
        \item {使用 \textbf{Ruby} 完成了后台 web 功能开发,体验了爽滑的开发体验。}
        \item {使用 \textbf{Rust} 重写了通信模块,优化了通信效率。}
        \item {使用 \textbf{React} 完成前端代码开发,搭配 TypeScript、Redux。}
        \item {为项目编写 Dockerfile,并使用 Docker 部署,使用 Jenkins 自动构建和部署项目,提升团队的开发效率。}
        \item {为项目引入 CheckStyle 插件,规范团队代码行为,保证代码的整体质量。}
      \end{cvitems}
    }

%---------------------------------------------------------
  \cventryjob
    {} % Job title
    {Carbon 语言设计} % Organization
    {1970.10-1970.01} % Location
    {} % Date(s)
    {
      \begin{cvitems} % Description(s) of tasks/responsibilities
        \item {\textbf{项目简介}:Carbon 从根本上说是一种后继语言方法,而不是逐步发展 C++ 的尝试。它是围绕与 C++ 的互操作性以及现有 C++ 代码库和开发人员的大规模采用和迁移而设计的,对惯用的 C++ 代码进行某种程度的源到源转换}
        \item {\textbf{技术选型}:Kotlin、Ruby、Rust、ReScript、Carbon。}
        \item {使用 Carbon 对原有的 C++ 代码进行重构,提升了项目的可读性}
        \item {使用 ReScript 替换原生的 JavaScript,是开发体验更流畅。}
        \item {重构后台所有功能,使用 Kotlin 重构 Java 代码的业务,开发体验提升。}
      \end{cvitems}
    }
    
      \cventryjob
    {} % Job title
    {Ruby-On-Rails} % Organization
    {2021.08-2022.12} % Location
    {} % Date(s)
    {
      \begin{cvitems} % Description(s) of tasks/responsibilities
        \item {\textbf{项目介绍}:Ruby 是一种解释型的面向对象的编程语言,通常用于 Web 开发。}
        \item {主要技术:Spring Boot、Maven、Docker、MySQL、MyBatis、Jacoco、JUnit、Shiro、Redis、Dubbo、Git}
        \item {使用 Rails 快速搭建项目骨架、使用 Generator 快速生成 CRUD 代码。}
        \item {使用 Go 开发通信模块。}
        \item {将项目拆分成多个模块,并使用 shell 脚本将项目微服务化。}
        \item {使用 JUnit 5 为接口编写单元测试与集成测试,并引入 Jacoco 生成代码测试覆盖率报告。}
      \end{cvitems}
    }

%---------------------------------------------------------
  \cventryjob
    {} % Job title
    {Scala 语言程序设计} % Organization
    {2021.08-2021.09} % Location
    {} % Date(s)
    {
      \begin{cvitems} % Description(s) of tasks/responsibilities
        \item {\textbf{项目介绍}: 如果你是 Haskell 的新手并且不确定从哪里开始,我们推荐CIS194。}
        \item {技术选型: Haskell、Scala 、Maven、Flyway、Junit 5、CircleCI、Git、GitHub}
        \item {使用 Haskell 用作主开发语言}
        \item {使用 RESTFul 风格开发接口、使用 Flyway 为数据库做迁移、版本管理等功能。}
        \item {使用 Junit 5 为整个模块进行单元测试。}
        \item {将项目接入 CircleCI,每次代码变更触发 CI 执行相应的测试、保证代码的高质量与稳定性。}
      \end{cvitems}
    }

\end{cventries}
